\chapter*{Agradecimientos}
\label{agradecimiento}

\normalsize Finalmente salió la Memoria y el título, la formalidad del evento nunca me preocupó mucho debido a mi pensamiento particular respecto de la elitización del conocimiento. Las universidades y grandes instituciones educacionales representan en el Chile de hoy un monumento a la privatización y receloso apoderamiento del conocimiento, entorpeciendo los procesos de aprendizaje de aquellos (la mayoría) que no tienen acceso a estas instituciones. El conocimiento es uno de los combustibles más poderosos para el desarrollo de la persona humana y nadie debiese entorpecer su libre circulación. No necesitas ir a una universidad para aprender, simplemente \textbf{mantener conciencia de la realidad, observar y escuchar lo que nos rodea}.\\

\normalsize Finalizar este trabajo, más que la obtención de un ''título'' o reconocimiento, representa el fin de una etapa, el último hito pendiente. Agradezco a mis viejos y familia por todo el apoyo incondicional que me dieron y el soporte para llegar donde estoy; a Eduardo por facilitar el camino; a Vicente que le tocó sentarse en el puesto de al lado para responder todas mis preguntas; al compañero Hernán, siempre con una solución práctica para todo; a los chiquillos de la pega por su alegría y buenas vibras; a Marcelo que me ''presto'' su planta de energía solar, a la profe Cecilia por su buena disposición y al profe Ricardo, que hace mucho tiempo me presento una forma diferente de apreciar la realidad.\\

\normalsize A todos los que defienden que la información y el conocimiento sea compartido, a quienes creen que la sociedad debe basarse en el \textbf{compartir} y no en la competencia.\\

\normalsize Al gran maestro, cerro, porque en esos viajes uno aprende sobre la vida, el universo, la naturaleza, las personas y sobre la \textbf{conciencia de ser humano}.\\

\normalsize Y no se me puede olvidar a todos los cabros que luchan, porque son ellos los que traen justicia y serán los que tenga paz al final del camino.

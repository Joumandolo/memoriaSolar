\chapter{Conclusiones}
\label{conclusiones}

\begin{itemize}
\item De acuerdo a los resultados obtenido en las comparaciones con los diferentes sistemas, se aprecian notables mejoras respecto del nuevo sistema de calculo. Los resultados son mejores de lo esperado y las diferencias se explican exclusivamente por la ausencia de la variable temperatura.
\item Uno de los parámetro mas influyentes en los cálculos teóricos, son las temperaturas de operación de la planta, por lo que se deja planteada la inquietud de implementar esta mejora en la calculadora ya que esto puede representar un significativo aumento en la calidad de los resultados finales.
\item El método de captura de datos y registro en la base de datos es poco robusto, ya que solo se establece una conversación unilateral entre la estación y la base de datos. Es necesario desarrollar un protocolo de comunicación bilateral con comparación de errores, además es necesario desarrollar un protocolo que permita consultar la estación y actualizar la base de datos asincronicamente, es decir poder actualizar la base de datos con los registros faltantes en cualquier momento dado, basándose en el registro interno del ''datalogger''.
\item La memoria del ''datalogger'' es volátil, esto quiere decir que si hay varios días sin sol, existe una gran posibilidad de que la estación quede sin energía y por ende pierda los datos. Para una mayor confiabilidad y resguardo de los datos, se debe implementar un sistema no volátil de almacenamiento en el mismo datalogger.
\item Durante el desarrollo de este trabajo se aprendió mucho sobre temas de energía, de sistemas solares, de Energía Renovables No Convencionales, de normas técnicas y de como debiese ser un buen sistema de producción de energía para un país. sin embargo, hay un tema muy relevante y que tienen que ver con el objetivo por el cual un país produce energía. Un país como Chile que se dice querer llegar a altos niveles de desarrollo en los próximos años, debe obligatoriamente tener esta cuestión resuelta, y lamentablemente estamos muy lejos de esto, se habla de que el consumo va a aumentar muy rápidamente en los próximos 10 años pero no se habla del porque este va a aumentar, nos enfocamos solo en producir mas energía sin un claro proyecto a largo plazo con el para que producir toda esta energía. Chile quiere se un país desarrollado, para eso necesita energía, pero esta energía no no ayuda a desarrollarnos si no pensamos en un plan que nos permita desarrollar la industria nacional en vez de continuar sobre-explotando los recursos naturales.
\item Si bien el uso de energías renovables es un gran aporte a disminuir el impacto ambiental que como sociedad causamos a nuestro entorno, el echo de reemplazar todas nuestras fuentes de energía no renovables por renovables, no nos asegura en lo mas mínimo un plan de desarrollo país sustentable mientras la energía producida se sigue utilizando para sobre-explotar los recursos naturales de manera indiscriminada lo que a largo plazo continuara por degradar el medio ambiente de la misma forma que hoy lo hacemos produciendo energías a partir de los fósiles.
\item Chile es el mejor lugar del mundo para producir energía solar, somos pioneros en experimentar con este tipo d energía en el mundo, sin embargo no hemos sido capaces de potenciar este desarrollo en nuestro propio país.
\end{itemize}

La presente consiste en el desarrollo de una plataforma de gestión de la información producida por la planta de energía solar de la exportadora Subsole S.A., ubicada en la región de Atacama.
Actualmente es la planta más grande de Chile para uso agrícola y el Banco Interamericano de Desarrollo (BID) está interesado en difundir características técnicas sobre el proceso de construcción y operación a tiempo real de esta instalación.
Dentro de este marco, el BID encargó a la Fundación Chile la creación de un portal Web junto a una Red de difusión y colaboración para Latinoamérica y el Caribe (RedSolLac), la cual pretende potenciar y desarrollar la producción y el uso de energía fotovoltaica en la región.\\

La información de energía generada por la planta e inyectada a la red es presentada en tiempo real por la plataforma Web, además queda disponible para ser descargada en diferentes formatos por los usuarios especialistas.\\

Para la captura y recopilación de datos se implementan estaciones meteorológicas compuestas de equipos especializados en la medición de variables solares, además de diferentes parámetros medioambientales. Se utilizan dos estaciones, una ubicada en la Región Metropolitana en la comuna de Vitacura en Fundación Chile y otra en la Región de Atacama en la comuna de Tierra Amarilla en la planta de Subsole.\\

A partir de la información obtenida por las estaciones, se desarrolla un sistema de cálculo para realizar estudios de prefactibilidad técnica y económica a través de una calculadora “on line”.\\
Los datos de las estaciones de medición y de y de la planta quedan a disposición de los usuarios de la RedSolLac en forma gráfica y con la posibilidad de ser descargados de la base de datos en diferentes formatos.

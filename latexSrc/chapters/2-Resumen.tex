La Memoria que se presenta a continuación tiene por objetivo desarrollar una plataforma de gestión de la información a partir de datos originados por estaciones meteorológicas solares y plantas de generación de energía fotovoltaicas. Esta plataforma se desarrolla a pedido de la ''Red Solar para Latinoamérica y el Caribe'' (RedSolLAC). La RedSolLAC es un proyecto encargado por el Banco Interamericano de desarrollo (BID) a la Fundación Chile. El BID a través de este proyecto pretende difundir, desarrollar y potenciar el uso de las Energía Renovables No Convencionales en la región, especialmente la producción y el uso de energía fotovoltaica.\\

La información recopilada durante el proceso de desarrollo de la plataforma, así como durante la operación, debe quedar a disposición de todos los usuarios de la RedSolLAC así como de la población en general, siendo presentada en tiempo real a través de la Web y además debe poder ser descargada en diferentes formatos por los usuarios especialistas.\\

Para la captura y recopilación de datos se utilizará una estación de medición ubicada en Fundación Chile, en la comuna de Vitacura, en la Región Metropolitana y una pequeña plata fotovoltaica residencial ubicada en la misma comuna. En la estación se implementaran equipos especializados en la medición de parámetros solares, además de otras variables medioambientales.\\

A partir de la información obtenida, se desarrollará un sistema de cálculo para realizar estudios de prefactibilidad técnica y económica a través de una calculadora “on line”.

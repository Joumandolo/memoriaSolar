La comuna de Peumo es una pequeña localidad  ubicada al norte de de la región del Libertador General Bernardo O'Higgins(VI Región) mas o menos a 30 minutos de Rancagua en vehículo. Es una zona campestre con una baja densidad de población y gran extinción de tierras dedicadas a la agricultura.\\

En este pequeño pueblo llamado Peumo nos encontramos con una organización de nombre Coopeumo que se dedica a apoyar a los pequeños agricultores de la zona tanto de la comuna de Peumo como de las comunas aledañas. Coopeumo es una cooperativa campesina que agrupa a mas de 400\cite{Peumo:Online} familias o pequeños agricultores que viven del trabajo de la tierra. La mayoría de estas familias no poseen grandes extinciones de terreno por lo que utilizan la cooperativa para lograr un nivel de producción elevado que les permita obtener un mejor precio en el mercado, lamentablemente quienes dominan la compra de la producción son muy pocas empresas lo cual lo convierte en un panorama difícil para los productores.\\

El año 2006 CTAC\gloss{Ctac:1} en conjunto con la Universidad Viña del Mar y la Municipalidad de Catemu concursan por un proyecto con Innova-Chile de CORFO para llevar a cabo la implementación de una red inalámbrica rural, Este proyecto resulto ser muy exitoso logrando obtener el premio nacional de inovacion tecnológica otorgado por CORFO. Durante el desarrollo de dicho proyecto se crea la primera \textbf{cooperativa tecnológica de infocomunicaciones Coopesic}\cite{Coopesic:Online}. El modelo de conectividad implementado por Coopesic en Catemu resulto ser muy eficiente por lo que posteriormente llevo a replicar la experiencia en diversas localidades entre la V y la VI Región del país, entre las cuales se encuentra la comuna de Peumo que gracias a una alianza entre Coopeumo, Coopesic y CTAC logran replicar el modelo de conectividad en la comuna y sus localidades cercanas logrando muy buena acogida por parte de la comunidad y de los socios de Coopeumo.\\

Este proyecto de conectividad rural permitió que muchas familias especialmente socios de la cooperativa lograran tener acceso a Internet y a infraestructura computacional a costos bastante reducido con un servicio de calidad, además permitió la interconectividad de la mayoría de las escuelas de comunas cercanas a Peumo.\\

El proyecto de implementación de un sistema de gestión agrícola nace bajo el alero del proyecto de conectividad. Con la tecnologías de conexión y comunicación implementada en las comunas cercanas se disponía de muchos recursos, los cuales debían ser aprovechados, es en este escenario en que la Cooperativa decide iniciar un proceso de renovación de sus sistemas de gestión y en conjunto con CTAC desarrollan un sistema de gestión agrícola basado en posicionamiento global(GPS) el cual permitiría a cada uno de los actores mejorar la eficiencia de sus procesos de gestión.\\

El sistema desarrollado permitiría a las familias productoras socias de la cooperativa administrar sus terrenos desde una plataforma virtual que les permitiría visualizar sus plantaciones mediante una imagen satelital y almacenar información relacionada con sus cultivos, recursos agrícolas y humanos. Al mismo tiempo permitiría a la cooperativa mantener información en tiempo real y en linea respecto de toda la producción de sus asociados, permitiendo mejorar el nivel de asesoramiento en la producción y generando planes de negocio que le permitirían mejoras sus rentabilidades, es importante mencionar que al hablar de mejorar sus utilidades y por ser una cooperativa esto representaba un incentivo para los mismo socios ya que podrían optar por ejemplo a recibir mejores pagos por sus cosechas.

The Report presented below aims to develop a platform for manage information of data resulting from weather stations and solar power generation plants. This platform is developed at the request of the ''Red Solar para Latinoamérica y el Caribe'' (Solar Network for Latin-America and the Caribbean). RedSolLAC is a project commissioned by the Inter-American Development Bank (IDB) to ''Fundación Chile''. Through this project IDB aims to disseminate, develop and promote the use of non-conventional renewable energy in the region, especially the production and use of photovoltaic power. \\

The information gathered during the plataform development process as well as during operation, it should be available to users of RedSolLAC the general population, being presented at real time on the Web and must also be able to be downloaded in various formats by specialist users. \\

For the capture and collection data will use a measuring station located in Santiago de Chile, within Fundacion Chile and also a small residential photovoltaic plant located in the same municipality. This station is implemented using specialized equipment in the measurement of solar parameters, and other environmental variables. \\

From the gathered information, it will develop a measurement system to study technical and economic feasibility through an on-line calculator. \\
